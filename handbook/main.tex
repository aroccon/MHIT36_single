\documentclass[11pt]{article}

\usepackage{sectsty}
\usepackage{graphicx}
\usepackage{amsmath, amssymb, amsfonts, amsthm}

% Margins
\topmargin=-0.45in
\evensidemargin=0in
\oddsidemargin=0in
\textwidth=6.5in
\textheight=9.0in
\headsep=0.25in
\DeclareMathOperator{\erf}{erf}
\DeclareMathOperator{\erfc}{erfc}

\title{MHIT36}
\author{Alessio Roccon}
\date{\today}

\begin{document}
\maketitle	
\pagebreak

% Optional TOC
% \tableofcontents
% \pagebreak

%--Paper--
\part{Formulation}

\section{Governing equations}

A first tentative try of the governing equation for boiling flow reads as [adapted from Karniadakis]:
\begin{equation}
\nabla \cdot  {\bf u} =  0\, ,
\end{equation}
%where $\rho_r=\rho_v/ \rho_l$.

For the Navier-Stokes equations, by assuming constant viscosity and density, the following equation can be derived:
\begin{equation}
\frac{\partial {\bf u}}{\partial t} + \nabla \cdot ({\bf u}{\bf u})= - \frac{ \nabla p}{\rho}  +  \nu \nabla^2 {\bf u}  + \frac{ \sigma}{\rho} k {\bf n} + {\bf f}
\end{equation}

The interface is captured using a second-order phase field method:
\begin{equation}
\frac{\partial \phi}{\partial t} +  \nabla \cdot ( {{\bf u} \phi }) = \nabla \cdot \left[ \gamma \left (\epsilon \nabla \phi  - \phi (1-\phi) \frac{\nabla \phi}{|\nabla \phi|}  \right) \right] \, ,
\end{equation}
The first two terms at the right hand side are the classical sharpening and diffusive terms.

Likewise, $\epsilon$ should be set equal to: 
\begin{equation}
\epsilon > 0.5 \Delta x
\end{equation}

\section{Numerical implementation}


\subsection{NS solver}

Projection-correction + Poisson solver based on 3D FFT along all directions


\subsection{CAC solver}

Euler explicit + FD2.
Solver is totally explicit 

\subsection{Forcing of turbulence}

When performing direct numerical simulations of homogeneous turbulence, one would like to force turbulence for two reasons. 
First, it permits to reach higher Reynolds numbers than in freely decaying turbulence. 
Second, under some assumptions, statistics can be obtained with time-averaging rather than ensemble averaging, which would be very costly considering the fact that refined statistics require a large number of samples. 
To study statistically stationary turbulence, many velocity forcing schemes have been used so far in numerical simulations. 
In homogeneous spectral simulations, large-scale forcing methods consist in providing energy to the low wavenumber modes, which is consistent with the concept of Richardson cascade. 
For example, considering a working an external force $\bf f$ which is a linear combination of sines and cosines with wavevectors of modulus lesser than or equal to $k_f$, the forcing contribution in both
Craya’s equation and energy spectral density equation vanishes at wave numbers greater than $k_F$.
Therefore, in statistically stationary turbulence, this external force feeds low wavenumbers and then part of this energy is transferred to larger wavenumbers through the nonlinear term.

\subsubsection{ABC forcing scheme}
The ABC forcing is defined as follows:
\begin{equation}
{\bf f} = [B \cos (k_F y) + C \sin (k_F z) ]{\bf i} \\ + [C \cos (k_F z)+ A \sin (k_F x) ]{\bf j} + [A \cos (k_F x) + B \sin (k_F y)] {\bf k}\, ,
\end{equation}
for a given large scale wavenumber $k_F$. 
Since ABC is an eigenfunction of the curl operator with eigenvalue $k_F$, the corresponding contribution in the helicity equation is nonzero and thus the ABC forcing injects helicity, in addition to energy, in the flow [See Intermittency in the isotropic component of helical and non-helical turbulent flows]
Usually $A=B=C=1$, other possible choice: $A=0.9$, $B=1$ and $C=0.9$, see paper listed above.

\subsubsection{TG forcing scheme}
The TG forcing is defined as follows:
\begin{equation}
{\bf f} = f_0 [ \sin (k_F x)  \cos (k_F y) \cos(k_F z) {\bf i}  - \cos (k_F x) \sin (k_F y) \sin (k_F z) {\bf j}] \, ,
\end{equation}
Here $f_0$ is the forcing amplitude, which can be set to have in the turbulent steady state all runs with r.m.s. velocities near unity.

\end{document}